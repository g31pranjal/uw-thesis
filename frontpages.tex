% T I T L E   P A G E
% -------------------

\pagestyle{empty}
\pagenumbering{roman}

\begin{titlepage}
        \begin{center}
        \vspace*{0.2cm}

        \Huge
        {\bf Integrating Column-Oriented Storage and Query
        	Processing Techniques into Graph Database
        	Management Systems }

        \vspace*{1.0cm}

        \normalsize
        by \\

        \vspace*{1.0cm}

        \Large
        Pranjal Gupta \\

        \vspace*{3.0cm}

        \normalsize
        A thesis \\
        presented to the University of Waterloo \\ 
        in fulfillment of the \\
        thesis requirement for the degree of \\
        Master of Mathematics \\
        in \\
        Computer Science \\

        \vspace*{2.0cm}

        Waterloo, Ontario, Canada, 2020 \\

        \vspace*{1.0cm}

        \copyright\ Pranjal Gupta 2020 \\
        \end{center}
\end{titlepage}

\pagestyle{plain}
\setcounter{page}{2}

\cleardoublepage

\cleardoublepage

% D E C L A R A T I O N   P A G E
% -------------------------------

\begin{center}\textbf{Author's Declaration}\end{center}

\noindent
I hereby declare that I am the sole author of this thesis. This is a true copy of the thesis, including any required final revisions, as accepted by my examiners.

\bigskip
  
\noindent
I understand that my thesis may be made electronically available to the public.

\cleardoublepage

% A B S T R A C T
% ---------------

\begin{center}\textbf{Abstract}\end{center}

Column-oriented RDBMSs, which support traditional read-heavy analytics workloads, employ a specific set of storage and query processing techniques for scalability and performance, such as positional tuple IDs, column-specific compression, and block-oriented processing. We revisit these techniques in the context of contemporary graph database management systems (GDBMSs). GDBMSs support a new set of analytics workloads, such as fraud detection in financial transaction networks or recommendations in social networks, that are also read-heavy but have fundamentally different access patterns than traditional analytics workloads. We first review the data characteristics and query access patterns in GDBMS to identify components of GDBMSs where existing columnar techniques can and cannot directly be used. We then present the physical data layout of columnar data structures, new columnar compression, and query-processing techniques that are optimized for GDBMSs. Our techniques include a new compact vertex and edge ID scheme, a new null and empty list compression scheme based on prefix-sums, and list-based query processing. We have integrated our techniques into GraphflowDB, an in-memory GDBMS. Compared to uncompressed storage, our compression techniques has scaled the system by 3.55x with minimal performance overheads. Our null compression scheme outperforms existing columnar schemes in query performance, with minor loss in compression rate and achieves both higher compression rate and better query performance as compared to row-oriented storage techniques adopted by existing GDBMSs. Finally, our list-based query processor techniques improve query performance by 2.7x on a variety of path queries and significantly outperform their corresponding conventional versions.

\cleardoublepage

% A C K N O W L E D G E M E N T S
% -------------------------------

\begin{center}\textbf{Acknowledgements}\end{center}

First and foremost, I want to thank my supervisor, Professor Semih Salihoglu, for all the guidance and support he has provided me over the past two years. Working with Semih has been a great experience and am really grateful to be a part of his research group. He has been a constant source of inspiration to whom I can look up to. Thank you for being a great teacher and guide.  I am incredibly fortunate to have been supervised by you and am excited about our future endeavours. 

I next want to thank Siddhartha, for being a friend to whom I can turn up anytime and for anything, and Amine, for helping me with the Graphflow project, collaborating with me on the A+ Indexes and guiding me at times whenever I needed help. Numerous all-nighters and breaks with you guys were fun. I am also grateful to Manoj, Antony and Aatish who helped me during my time here; and to my friends back in India, Yash and Garvit, with whom I've have done long video calls and had directionless conversations. I would also like to thank Siddhartha, Amine and Xiyang for proofreading my thesis.

I would also like to thank my parents for their guidance, support and encouragement in all my endeavours in life. Thank you for your trust in me and for allowing me to discover my dreams- that's all I ever desired. I also can't go without thanking my sister, Pranshri, and brother, Shrijal, for keeping up my morale. 

Finally, I want to thank my friend, Snehal, for being my constant source of love, happiness and motivation. Thank you for seeing me through all my good and bad phases. Her phone calls, gestures, and constant sweet interference in my life has helped me cope up the direst of fears and come out good.

\cleardoublepage

% D E D I C A T I O N
% -------------------

\begin{center}\textbf{Dedication}\end{center}

This is dedicated to the ones I care about .
\cleardoublepage

% T A B L E   O F   C O N T E N T S
% ---------------------------------
\renewcommand\contentsname{Table of Contents}
\tableofcontents
\cleardoublepage
\phantomsection    % allows hyperref to link to the correct page

% L I S T   O F   F I G U R E S
% -----------------------------
\addcontentsline{toc}{chapter}{List of Figures}
\listoffigures
\cleardoublepage
\phantomsection		% allows hyperref to link to the correct page

% L I S T   O F   T A B L E S
% ---------------------------
\addcontentsline{toc}{chapter}{List of Tables}
\listoftables
\cleardoublepage
\phantomsection		% allows hyperref to link to the correct page

% GLOSSARIES (Lists of definitions, abbreviations, symbols, etc. provided by the glossaries-extra package)
% -----------------------------
\printglossaries
\cleardoublepage
\phantomsection		% allows hyperref to link to the correct page

% Change page numbering back to Arabic numerals
\pagenumbering{arabic}

