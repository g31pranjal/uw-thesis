\documentclass[letterpaper,12pt,titlepage,oneside,final]{book}
 
% For PDF, suitable for double-sided printing, change the PrintVersion variable below
% to "true" and use this \documentclass line instead of the one above:
%\documentclass[letterpaper,12pt,titlepage,openright,twoside,final]{book}

% Some LaTeX commands I define for my own nomenclature.
% If you have to, it's better to change nomenclature once here than in a 
% million places throughout your thesis!
\newcommand{\package}[1]{\textbf{#1}} % package names in bold text
\newcommand{\cmmd}[1]{\textbackslash\texttt{#1}} % command name in tt font 
\newcommand{\href}[1]{#1} % does nothing, but defines the command so the
    % print-optimized version will ignore \href tags (redefined by hyperref pkg).
%\newcommand{\texorpdfstring}[2]{#1} % does nothing, but defines the command
% Anything defined here may be redefined by packages added below...

% This package allows if-then-else control structures.
\usepackage{ifthen}
\newboolean{PrintVersion}
\setboolean{PrintVersion}{false} 
% CHANGE THIS VALUE TO "true" as necessary, to improve printed results for hard copies
% by overriding some options of the hyperref package below.

%\usepackage{nomencl} % For a nomenclature (optional; available from ctan.org)
\usepackage{amsmath,amssymb,amstext} % Lots of math symbols and environments
\usepackage[pdftex]{graphicx} % For including graphics N.B. pdftex graphics driver 

% Hyperlinks make it very easy to navigate an electronic document.
% In addition, this is where you should specify the thesis title
% and author as they appear in the properties of the PDF document.
% Use the "hyperref" package 
% N.B. HYPERREF MUST BE THE LAST PACKAGE LOADED; ADD ADDITIONAL PKGS ABOVE
\usepackage[pdftex,pagebackref=false]{hyperref} % with basic options
		% N.B. pagebackref=true provides links back from the References to the body text. This can cause trouble for printing.
\hypersetup{
    plainpages=false,       % needed if Roman numbers in frontpages
    unicode=false,          % non-Latin characters in Acrobat’s bookmarks
    pdftoolbar=true,        % show Acrobat’s toolbar?
    pdfmenubar=true,        % show Acrobat’s menu?
    pdffitwindow=false,     % window fit to page when opened
    pdfstartview={FitH},    % fits the width of the page to the window
    pdftitle={Integrating\ Column-Oriented\ Storage\ and\ Query\ Processing\ Techniques\ Into\ Graph\ Database\ Management\ Systems},
    pdfauthor={Pranjal\ Gupta},
	pdfsubject={Computer\ Science},
	pdfkeywords={graph\ database} {database\ system} {storage} {compression} {adjacency\ lists} {columnar-storage}
    pdfnewwindow=true,      % links in new window
    colorlinks=true,        % false: boxed links; true: colored links
    linkcolor=blue,         % color of internal links
    citecolor=green,        % color of links to bibliography
    filecolor=magenta,      % color of file links
    urlcolor=cyan           % color of external links
}

\ifthenelse{\boolean{PrintVersion}}{   % for improved print quality, change some hyperref options
\hypersetup{	% override some previously defined hyperref options
%    colorlinks,%
    citecolor=black,%
    filecolor=black,%
    linkcolor=black,%
    urlcolor=black}
}{} % end of ifthenelse (no else)

\usepackage[automake,toc,abbreviations]{glossaries-extra} % Exception to the rule of hyperref being the last add-on package
% If glossaries-extra is not in your LaTeX distribution, get it from CTAN (http://ctan.org/pkg/glossaries-extra), 
% although it's supposed to be in both the TeX Live and MikTeX distributions. There are also documentation and 
% installation instructions there.

% Setting up the page margins...
% uWaterloo thesis requirements specify a minimum of 1 inch (72pt) margin at the
% top, bottom, and outside page edges and a 1.125 in. (81pt) gutter
% margin (on binding side). While this is not an issue for electronic
% viewing, a PDF may be printed, and so we have the same page layout for
% both printed and electronic versions, we leave the gutter margin in.
% Set margins to minimum permitted by uWaterloo thesis regulations:
\setlength{\marginparwidth}{0pt} % width of margin notes
% N.B. If margin notes are used, you must adjust \textwidth, \marginparwidth
% and \marginparsep so that the space left between the margin notes and page
% edge is less than 15 mm (0.6 in.)
\setlength{\marginparsep}{0pt} % width of space between body text and margin notes
\setlength{\evensidemargin}{0.125in} % Adds 1/8 in. to binding side of all 
% even-numbered pages when the "twoside" printing option is selected
\setlength{\oddsidemargin}{0.125in} % Adds 1/8 in. to the left of all pages
% when "oneside" printing is selected, and to the left of all odd-numbered
% pages when "twoside" printing is selected
\setlength{\textwidth}{6.375in} % assuming US letter paper (8.5 in. x 11 in.) and 
% side margins as above
\raggedbottom

% The following statement specifies the amount of space between
% paragraphs. Other reasonable specifications are \bigskipamount and \smallskipamount.
\setlength{\parskip}{\medskipamount}

% The following statement controls the line spacing.  The default
% spacing corresponds to good typographic conventions and only slight
% changes (e.g., perhaps "1.2"), if any, should be made.
\renewcommand{\baselinestretch}{1} % this is the default line space setting

% By default, each chapter will start on a recto (right-hand side)
% page.  We also force each section of the front pages to start on 
% a recto page by inserting \cleardoublepage commands.
% In many cases, this will require that the verso page be
% blank and, while it should be counted, a page number should not be
% printed.  The following statements ensure a page number is not
% printed on an otherwise blank verso page.
\let\origdoublepage\cleardoublepage
\newcommand{\clearemptydoublepage}{%
  \clearpage{\pagestyle{empty}\origdoublepage}}
\let\cleardoublepage\clearemptydoublepage

% Define Glossary terms (This is properly done here, in the preamble. Could be \input{} from a file...)
% Main glossary entries -- definitions of relevant terminology
\newglossaryentry{computer}
{
name=computer,
description={A programmable machine that receives input data,
               stores and manipulates the data, and provides
               formatted output}
}

% Nomenclature glossary entries -- New definitions, or unusual terminology
\newglossary*{nomenclature}{Nomenclature}
\newglossaryentry{dingledorf}
{
type=nomenclature,
name=dingledorf,
description={A person of supposed average intelligence who makes incredibly brainless misjudgments}
}

% List of Abbreviations (abbreviations type is built in to the glossaries-extra package)
\newabbreviation{aaaaz}{AAAAZ}{American Association of Amature Astronomers and Zoologists}

% List of Symbols
\newglossary*{symbols}{List of Symbols}
\newglossaryentry{rvec}
{
name={$\mathbf{v}$},
sort={label},
type=symbols,
description={Random vector: a location in n-dimensional Cartesian space, where each dimensional component is determined by a random process}
}
 
\makeglossaries


\begin{document}

% T I T L E   P A G E
% -------------------

\pagestyle{empty}
\pagenumbering{roman}

\begin{titlepage}
        \begin{center}
        \vspace*{0.2cm}

        \Huge
        {\bf Integrating Column-Oriented Storage and Query
        	Processing Techniques into Graph Database
        	Management Systems }

        \vspace*{1.0cm}

        \normalsize
        by \\

        \vspace*{1.0cm}

        \Large
        Pranjal Gupta \\

        \vspace*{3.0cm}

        \normalsize
        A thesis \\
        presented to the University of Waterloo \\ 
        in fulfillment of the \\
        thesis requirement for the degree of \\
        Master of Mathematics \\
        in \\
        Computer Science \\

        \vspace*{2.0cm}

        Waterloo, Ontario, Canada, 2020 \\

        \vspace*{1.0cm}

        \copyright\ Pranjal Gupta 2020 \\
        \end{center}
\end{titlepage}

\pagestyle{plain}
\setcounter{page}{2}

\cleardoublepage

\cleardoublepage

% D E C L A R A T I O N   P A G E
% -------------------------------

\noindent
I hereby declare that I am the sole author of this thesis. This is a true copy of the thesis, including any required final revisions, as accepted by my examiners.

\bigskip
  
\noindent
I understand that my thesis may be made electronically available to the public.

\cleardoublepage

% A B S T R A C T
% ---------------

\begin{center}\textbf{Abstract}\end{center}

Column-oriented RDBMSs, which support traditional read-heavy analytics workloads, employ a set of storage and query processing techniques for scalability and performance, such as positional tuple IDs, column-specific compression, and block-oriented processing. We revisit these techniques in the context of contemporary graph database management systems (GDBMSs). GDMSs support a new set of analytics workloads, such as fraud detection in financial transaction networks or recommendations in social networks, which are also read-heavy but have fundamentally different access patterns than traditional analytics workloads. We first review the properties of data and access patterns of queries in GDBMS to identify the components of GDBMSs where existing columnar techniques can and cannot directly be used. We then present the physical data layout of columnar data structures, new columnar compression, and query-processing techniques that are optimized for GDBMSs. Our techniques include a new compact vertex and edge ID scheme, a new null and empty list compression scheme based on prefix-sums, and list-based query processing. We have integrated our techniques into GraphflowDB, an in-memory GDBMS. Compared to uncompressed storage, our compression techniques has scaled the system by Xx with minimal performance overheads. Our null compression scheme outperforms existing columnar schemes in query performance, with minor loss in compression rate and achieves both higher compression rate and better query performance as compared to row-oriented storage techniques adopted by existing GDBMSs. Finally, our list-based query processor techniques improve query performance by Xx on a variety of path queries and significantly outperform their corresponding conventional versions.

\cleardoublepage

% A C K N O W L E D G E M E N T S
% -------------------------------

\begin{center}\textbf{Acknowledgements}\end{center}

First and foremost, I want to thank my supervisor, Professor Semih Salihoglu, for all the guidance and support he has provided me over the past two years. Working with Semih has been a great experience and am really grateful to be a part of his reaserch group. He has been a constant source of inspiration to whom I can look upto. Thank you for being a great teacher and guide.

I next want to thank Siddhartha, for being a friend to whom I can turn up anytime and for anything, and Amine, for helping me with the Graphflow project, collaborating with me on the A+ Indexes and also coping with my not-so-good reaction at times. All nighters were fun too with you guys. I am also grateful to Manoj, Antony and Aatish who helped me during my time here; and to my friends Yash and Garvit for just being there to listen to me.

I also want to express my sincerest gratitude to my thesis readers, Ken Salem and Tamer Özsu, for taking out valuable time from their busy schedules to read my thesis and provide their valuable comments. 

I would also like to thank my parent for their guidence, support and encouragement in all my endeavours in life; and my siblings, Pranshri and Shrijal, for keeping up my morale. 

Finally, I want to thank my friend, Snehal, for being my constant source of love, happiness and motivation. Thank you for seeing me through all my good and bad phases. 

\cleardoublepage

% D E D I C A T I O N
% -------------------

\begin{center}\textbf{Dedication}\end{center}

This is dedicated to the ones I care about .
\cleardoublepage

% T A B L E   O F   C O N T E N T S
% ---------------------------------
\renewcommand\contentsname{Table of Contents}
\tableofcontents
\cleardoublepage
\phantomsection    % allows hyperref to link to the correct page

% L I S T   O F   T A B L E S
% ---------------------------
\addcontentsline{toc}{chapter}{List of Tables}
\listoftables
\cleardoublepage
\phantomsection		% allows hyperref to link to the correct page

% L I S T   O F   F I G U R E S
% -----------------------------
\addcontentsline{toc}{chapter}{List of Figures}
\listoffigures
\cleardoublepage
\phantomsection		% allows hyperref to link to the correct page

% GLOSSARIES (Lists of definitions, abbreviations, symbols, etc. provided by the glossaries-extra package)
% -----------------------------
\printglossaries
\cleardoublepage
\phantomsection		% allows hyperref to link to the correct page

% Change page numbering back to Arabic numerals
\pagenumbering{arabic}

 


\chapter{Introduction}
\label{introduction}

The term \gls{gdbms} in its contemporary usage refers to data management software, such as Neo4j \cite{neo4j}, JanusGraph~\cite{janusgraph}, TigerGraph~\cite{tigergraph}, and GraphflowDB~\cite{kankanamge:graphflow, mhedhbi:sqs}, that adopt the property graph data model~\cite{neo4j-property-graph-model}. \gls{gdbms}s have lately gained popularity among a wide range of analytics applications, such as fraud detection and risk assessment in financial services to recommendations in e-commerce and social networks~\cite{sahu:survey}. These applications have read-heavy workloads that search for patterns in a graph-structured database, which often requires processing large amounts of data. In the context of \gls{rdbms}, column-oriented storage ({\em column stores})~\cite{c-store, monetdb, vectorwise, oracle, ziauddin:zone-maps, ms} employ a set of storage, indexing, and query processing techniques to support traditional read-heavy analytics applications, such as business intelligence and reporting, that also process large amounts of data. As such, these columnar techniques are relevant for improving the performance and scalability of Gs.

In this theis, we revisit several columnar storage and query processing techniques and investigate their integration into GDBMSs. Specifically, we discuss the applicability of columnar storage techniques, compression schemes for columns~\cite{}, block-oriented query processing~\cite{}, and zone maps~\cite{}, for storing and accessing different components of GDBMSs. 

We first identify the cases when GDBMS can directly integrate these techniques. For example, columnar storage is directly applicable for storing vertex properties. Similarly, the popular compressed sparse row (CSR) or column (CSC) formats to store the topology of the graphs, i.e., the edges between vertices, are columnar data structures that employ a variant of run-length encoding compression scheme. We then identify the cases when there are significant differences between column stores and GDBMSs that require redesigning some these techniques in the context of GDBMSs. For example, we show the inefficiency of using straightforward positional edge IDs and using columns for edge properties in the context of GDBMSs. Similarly, we show that existing null compression schemes from column stores would lead to very slow access in the context of GDBMSs. We then redesign new GDBMS-specific versions of these techniques. For example, we describe a new prefix-sum based null compression scheme, which is broadly applicable to suppress both null vertex properties as well as empty adjacency lists. After appropriate redesigning, we show that these techniques can significantly improve both the scalability and the performance of GDBMSs.

\section{Contributions of the thesis}

The specific contributions of this paper are as follows:

\begin{itemize}
	\item {\bf General Guidelines:} We begin by overviewing the properties of data access patterns in GDBMSs, from which we derive a set of general guidelines for designing the physical data layout of \gls{gdbms}s. These guidelines instruct where columnar techniques are possible to use and some fundamental limitations about which type of data accesses can be localized in GDBMSs without data replication.
	
	
	
	\item {\bf Columnar Storage:} In Section~\ref{sec:columnar-storage}, we explore  the application of columnar data structures for storing different components of the databases of GDBMSs. We start with components that can directly be stored in columnar structures, such as vertex properties, and one-to and many-to-many edges. We then discuss how to use columnar structures for edge properties, which requires designing positional edge IDs. We start by discussing two natural edge ID schemes and their corresponding columnar structures and discuss their shortcomings. We then describe a new scheme we call {\em page-level positional IDs} and its corresponding {\em vertex-group CSR structure} which addresses these shortcomings. We end this section with a decision tree for deciding which columnar structures are suitable for storing edges and edge properties.  
	\item {\bf Columnar Compression:} In Section~\ref{sec:columnar-compression}, we discuss the application of columnar compression techniques in GDBMSs. We start by reviewing directly applicable compression techniques. We then review existing null compression techniques and their shortcomings and describe a new null compression scheme, based on storing prefix sums, that address these shortcomings. Our scheme allows constant time access to any null or non-null property and requires 2 bits per stored element. Finally, we describe the application of using run-length encoding for storing the page-level positional edge IDs we describe in Section~\ref{sec:columnar-storage}. 
	\item {\bf List-oriented Processing:} In Section~\ref{sec:list-oriented-processing}, we review block- or vector-oriented query processors employed by several column stores and discuss their shortcoming in the context of GDBMSs. We then introduce a new list-based block-oriented query processor, which uses both Volcano-style processing, where single tuples are passed between operators, as well as block-oriented processing, where blocks are variable size and based on the sizes of the lists that need to be processed. 
	
	\item {\bf List-based Zone Maps:} In Section~\ref{sec:list-maps}, we revisit the zone-map~\cite{} indexes that  is employed in some column stores. Zone maps store aggregates, such as min, sum, or max, on each page of a column, which are used by the system for both pruning the page when queries have predicates or performing an aggregate. We discuss the limitation of this approach in the context of GDBMSs, and adapt it to GDBMSs by defining list-based zone maps, which store these aggregates per list.
	
	\item Section~\ref{sec:evaluations} presents our evaluations. We show that integrating these columnar techniques into an in-memory GDBMS increases the systems' scalability by Xx and performance by Yx. We also show that our GDBMS-specific techniques outperform traditional columnar techniques significantly on micro-benchmarks as well as the popular LDBC bechnmark.
	
	\item Finally, Sections~\ref{sec:rw} and~\ref{sec:conclusions}, respectively, review related work and conclude.   
\end{itemize}


\chapter{General Guidelines for Optimizing the Physical Data Layout and Query Processor in GDBMSs}
\label{c:guidelines}

In this chapter, we review the primary components that form the storage layers of \gls{gdbms}s, system's primary operators and the general data access patterns of operators when evaluating a query. We then draw a basic set of guidelines that will instruct the design of physical data layout and query-processing techniques introduced in later chapters.

Section \ref{sec:property-graph-data-model} briefly describes the \emph{property graph data model}. Section \ref{sec:storage-components} describes the primary storage components of \gls{gdbms}s that adopt the property graph data model, while Section \ref{sec:operators} reviews the query processing operators in \gls{gdbms}s. We end the chapter by stating our guidelines in Section \ref{sec:guidelines}.

\section{Property Graph Data Model}
\label{sec:property-graph-data-model}

\begin{figure}
	\includegraphics[scale=0.86]{img/property-graph}
	\vspace{-8pt}
	\caption{Running example graph.}
	\label{fig:runn}
	\vspace{-8pt}
\end{figure}

Figure~\ref{fig:runn} shows a graph data in the property graph data model. A property graph consists \emph{vertices} that represent entities and directed \emph{edges} between vertices that represent relationships between entities. Each vertex and edge has a particular \emph{label}, describing the high-level categories of vertices and edges. For example, in Fig.~\ref{fig:runn}, vertices have labels: \texttt{PERSON} and \texttt{ORG}, while edges have labels: \texttt{FOLLOWS}, \texttt{WORKAT} and \texttt{STUDYAT}.

Similar to columns in relational tables, vertices and edges can have (key, value) \emph{properties}. Although the properties of vertices and edges do not need to adhere to a strict \emph{schema }, in practice many of these properties will be highly structured, i.e., similar set of properties will exist on vertices and edges of the same label.

\section{Primary Storage Components in GDBMSs}
\label{sec:storage-components}

In every \gls{gdbms} we are aware of, the edges of a graph are stored in data structure called \emph{adjacency lists}. An adjacency list of a vertex is a map to a vertex's direct set of \emph{adjacent} edges and \emph{neighbouring} vertices. In a \gls{gdbms}, each vertex has 2 adjacency lists - a \emph{forward adjacency list} containing outward edges of that vertex, and a \emph{backward adjacency list} that holds inward edges of the vertex. One can think of edges in the graph as a relational table, having 3 attributes, a source vertex, a destination vertex and an \texttt{edgeID}. The adjacency lists can then be thought of as an \emph{index} on this relational table that is \emph{clustered} by either the source or destination vertex. In practice, often this index has depth of 1 or 2, so given a vertex, a system can access its list of edges in 1 or 2 lookups. By having adjacency lists of a vertex $u$ in either direction, the system can access the list of outward and inward edges and neighbouring vertices of $u$ directly in a constant-time lookup operation, which provides the core capability of fast joins to a \gls{gdbms}. 

Typically, a single-directional adjacency list of vertex $u$ is further clustered into sublists by the edge label of the edges. This enables extending a vertex through its edges with a \emph{particular} label in constant time. The rationale behind the sub-clustering is that many queries by applications have specific label on query edges. Some systems further order the edge in the adjacency lists either by a property of adjacent edge or neighbour vertex or simply by neighbour vertex. Sorting enables the system to access parts of list in time logarithmic in the size of the adjacency list.

A \gls{gdbms} also stores properties that appear on the vertices and the edges. There are multiple solutions for storing properties. The most straightforward approach is to store properties in a \emph{key-value store} \cite{dgraph} and referenced by the attribute key and the \texttt{vertexID} or \texttt{edgeID}. Properties can also be stored as a \emph{variable-sized} byte-encoded record of each vertex or edge in the same manner as row-oriented \gls{rdbms} stores tuples. A record is considered variable-sized because the number of properties on an entity is not fixed. Searching for a property in variable-sized records involves decoding and parsing the entire record until the matching attribute is found, which can be very slow. Also, the addition and deletion of properties are not straightforward in records. Yet another way of storing properties is in a doubly linked-list, as in Neo4j \cite{neo4j}, that makes additions and deletions easy though searching is still a linear-time operation.

\section{Query Execution in GDBMSs}
\label{sec:operators}

In this section, we review the general execution of queries in a \gls{gdbms} by analyzing major operators used in the query plans. Though systems differ in their architectures and implementation of operators that they support, there still remains a similarity in their data access patterns. We use the Cypher query language \cite{cypher} to describe the queries to a \gls{gdbms}. A user query typically consists of 3 parts, 1) a \texttt{MATCH} clause with a subgraph query pattern $Q(V_Q, E_Q)$, where $V_Q$ and $E_Q$ are the query vertex and edge variables, that is matched to the input graph's topology; 2) \texttt{WHERE} claue that contains a predicate $\rho$ over properties of edge and vertex variable that the matched subgraph must satisfy; and 3) a \texttt{RETURN} statement that returns a projection of matches in the graph or its aggregated value. Example \ref{ex:cypher-example} shows a typical query written in Cypher language, to query the example graph in figure~\ref{fig:runn}.

\begin{example}
	\label{ex:cypher-example}
	Consider the following query Q. 
	{\em 
		\begin{lstlisting}[numbers=none,  showstringspaces=false,belowskip=0pt ]
		MATCH (a:PERSON)$-$[e:WORKAT]$\rightarrow$(b:ORG)
		WHERE a.age $>$ 22 AND b.estd < 2015
		RETURN *\end{lstlisting}
	}
	This query returns all the PERSON vertices and their workplaces, constrained to the condition that the \textsc{\char13}\texttt{age}\textsc{\char13} property of PERSON vertex has a value that is greater than 22 and \textsc{\char13}\texttt{established}\textsc{\char13} property of ORG vertex is less than 2015. a and b are query vertex variables while e is a query edge variable.
\end{example}
\vspace{-5pt}

The following are the major operators used for matching the subgraph pattern and evaluating predicates in a query.


\begin{itemize}
	
	\item \textbf{\texttt{SCAN}}: Scans a set of vertices and edges from the graph topology.
	
	\item \textbf{\texttt{NEIGHBOURHOOD JOIN}}: e.g. \texttt{EXTEND/INTERSECT} in Graphflow, \texttt{EXPAND} in Neo4j. On a high level, the neighbourhood join operator matches the subgraph query pattern $Q$, one vertex variable at a time. The input to the operator is a partial $k$-match, $t$, of $Q$. We define a partial $k$-match of $Q$ as a set of vertices from the input graph assigned to the projection of $Q$ onto the set of $k$ query vertex variables. For each partially matched $t$, the operator extends $t$ by matching an unmatched query vertex, say $v$, such that there is atleast one edge in $E_Q$ between $v$ and $t$'s vertices' variables. The \emph{join} happens by sequentially reading adjacent edges and neighbour vertices from (forward/backward) adjacency list of one or more matched vertices of $t$, to produce a $k+1$-match.

	\item \textbf{\texttt{PROPERTY READER}}: (Vertex/Edge) property reader reads a property value of any vertex or edge that has been assigned to a variable in $V_Q$ or $E_Q$ of a partial match $t$, from the underlying properties storage. 

	\item \textbf{\texttt{FILTER}}: Given the predicate $\rho$ from the \texttt{WHERE} clause of the query and a partial match, $t$, of $Q$, the \texttt{FILTER} operator omits $t$ from the result of the query if $t$ does not pass the predicate $\rho$.
	
\end{itemize}

\begin{figure}
	\hfill\includegraphics[scale=0.80]{img/ex-qp}\hfill
	\vspace{-8pt}
	\caption{Query plan for Example~\ref{ex:cypher-example}.}
	\label{fig:ex-qp}
	\vspace{-8pt}
\end{figure}

Figure~\ref{fig:ex-qp} shows one of the query plans that the system will generate to execute query in Example~\ref{ex:cypher-example}. It consists of the following sequence of operators: 1) \texttt{SCAN} operator that matches the variable $a$ in query to vertex in the graph having label \texttt{PERSON}, 2) \texttt{JOIN} operator matches $b$ by reading the forward adjacency list of $a$'s match, 3) \texttt{PROPERTY READER} reads the properties \texttt{age} and \texttt{estd} of $a$'s and $b$'s match, and 4) \texttt{FILTER} operators filters out the matched query pattern that do not confrm to the constraint $a.age > 22$ $AND$ $b.estd < 2015$.

\section{Guidelines}
\label{sec:guidelines}

We next outline a set of guidelines for designing the physical data layout and query processor of a \gls{gdbms}.

\begin{guideline}[Edges are doubly-indexed.]
\vspace{-5pt}
Each edge appears in the forward adjacency list of that edge's source vertex and the backward adjacency list of it's destination vertex. This results in a 2x replication factor in storing the topology of a graph in the system. This  replication cannot be avoided by dropping an adjacency list in any one direction without hampering the capability to perform fast neighbour joins, which is one of the core feature of \gls{gdbms}s.
\end{guideline}

\label{ssec:edges-ordered}
\begin{guideline}[Edge properties are read in the same order as edges in an adjacency list.]
During the execution of a query, the \texttt{JOIN} operator will access the edges of a vertex $v$ in the order these edges appear in $v$'s (forward/backward) adjacency lists. Note that the edge or its property might not be read in consecutive operations, hence we do not call the access \emph{sequential}; nevertheless, ordering is preserved. If the query also needs to access the properties of these edges, the access to these edge properties will also be in the same order in which edges were read from the adjacency list. 

This raises the question of whether we can also read edges and edge properties sequentially from the memory at once and hence benefit from the cache locality available at hand. This can be achieved by organizing the edge propreties into lists in an order identical to that of edges in the adjacency list. Since, each edge appears twice, each edge property has to be replicated twice in the system. This might be quite expensive because every property and not just the edge and neighbour vertex need to be replicated. In addition, if the edges in the adjacency lists are sorted, as is done in some of the systems, then the list of ordered edge properties also needs to be kept in sorted order. This makes updates to the system costly, since an insertion of a new edge in between the adjacency list will effect the ordering of all the edge's property lists.

\end{guideline}

\begin{guideline}[Vertices cannot be ordered to make access from all neighbour verteices sequential.]
\label{gdln:vertices-unordered}
\vspace{-5pt}
Contrary to how the edges and edge properties can be strictly ordered for each of the adjacency lists, there cannot be an ordering on the vertices that localizes the access to neighbour vertices of every vertex and to the properties of these neighbour vertices without prohibitive data replication. In general, if a vertex $v$ has $n$ neighours, then $v$ and its properties need to be replicated $n$ times. Hence, localizing access to neighbour vertices and their properties should not be put in the desiderata of the system's physical data layout design.

\end{guideline}

\begin{guideline}[Graph data often has partial structure.]
\label{gdln:graph-schema}
Even though the property graph data model is semi-structured, in practise many graph databases stored in \gls{gdbms}s have structure in different components, which \gls{gdbms}s can exploit. One reason this structure exist is that, as observed by prior works \cite{survey}, often the data in \gls{gdbms} comes from structured data in \gls{rdbms}. Infact, several of the \gls{gdbms} providers from the industry and some academics are actively working of defining a schema language for the property graph data model \cite{schema-validation-bonifati, defining-schema-hartig}. We identify three commonly appearing structure in property graph data:


\begin{enumerate}
	
	\item Often, edge labels in the graph data have a well-defined set of source and destination vertex label(s). This restricts the vertices to  having inward or outward edges of only a definite set of labels. In our example graph, edges having label FOLLOWS only exists between vertices of label PERSON.
	
	\item The number of edges of a particular label to which a source or destination vertex can be associated, is a property of the edge label. We call this the \emph{cardinlaity} of an edge label. \emph{One-to-one} cardinality means that there can only be single edge of a particular label from a source vertex and to a destination vertex. \emph{Many-to-one} permits a single edge of a label from a source vertex but multiple edges to a destination vertex. Similar analogy can be applied to \emph{one-to-many} and \emph{many-to-many} cardinality edge labels too.
	
	\item Similar to the attributes of a relational table, properties on an edge or vertex and the datatypes of these properties can \emph{often} be determined by its edge or vertex label. In our example graph, all vertices having label PERSON have 3 properties: name:\texttt{STRING}, age: \texttt{INT} and gender:\texttt{STRING}. As long as a significant fraction of vertices and edges with a particular label have a common set of properties, a system can exploit this structuredness to store these properties more efficiently. 
	
\end{enumerate}


Such structure in data provides an opportunity to design more efficient and simpler data structures for accessing the storage layer of \gls{gdbms}. However, not all data in graph databases has structure. As a working terminology, we will use the following terms:


\begin{itemize}
	\item \textbf{Structured/unstructured edge:} An edge of a particular label, that follows above-mentioned points \ref{gdln:graph-schema-rule1} and \ref{gdln:graph-schema-rule2}, is called an structured edge. An edge that is not structured, is called an unstructured edge.
	 
	\item \textbf{Structured/unstructured property:} A structured property is a property on a vertex or edge that, 1) can be determined by the vertex type or edge label of that entity; 2) appears in a significant fraction of the vertices of edges of a particular label; and 3) have the same data type in all its occurrence. Any property that is not a structured poperty, is considered unstructed.
	
\end{itemize}


We focus on optimizing the storage of structured part of the graph data in this thesis. It forms an interesting research topic to optimize a system for unstructured part of graph data. A standard approach is to serialize the key, datatype and value of each property  of our work on structured properties. Owing to the erratic nature of unstructured properties, storing them in variable-length records or linked-list as (attribute, value) pairs is a viable solution. Structured property storage can, however, be optimized to benefit memory footprint as well as access performance.

\end{guideline}

\begin{guideline}[Queries read a small subset of the vertex or edge properties]
In order to understand the nature of queries user ask on \gls{gdbms}, we conducted a survey of 100 StackOverflow questions containing openCypher queries. We focused on queries of analytical nature and discards transactional ones like insert, delete and update. We observe the following: 


\begin{itemize}
	
	\item Out of the 100 queries, 68 accessed at least one of the properties on a vertex or an edge. Of these, 61 accesses vertex properties and 13 accesses edge properties.
	
	\item Only 11 queries returned all the properties of a query edge or vertex, while 35 of them return specific properties.
	
	\item Average number of properties accessed by those queries that explicitly return a set of properties is only 1.6.
	
\end{itemize}

We can observe that conclude that vertex properties are more popularly accessed as compared to edge properties and most of the queries only access 1 or at most 2 properties.

\end{guideline}


\chapter{Columnar Storage}
\label{c:columnar-storage}

In this section, we explore the application of columnar data structures for different storage components of the \gls{gdbms}. 

Section \ref{sec:vertex-property-columns} describes the design of columns for vertex properties. In section \ref{sec:edge-property-columns}, we state the limitations of using sub-optimal approaches of storing edge properties in columns. We introduce \emph{edge property lists} for storing edge properties in a way efficient for storage and access. The aforementioned sections also describe a new and more compact identification scheme for vertex and edges in the system. Our new identification scheme replace 8-byte \texttt{vertexID}s and \texttt{edgeIDs}. Section \ref{sec:adjacency-lists} describes adjacency lists in light of columnar storage. We finally end by discussing several storage optimizations that can be done on our new data structures and identification scheme in order to reduce the memory footprint.

\section{Columns for Vertex properties}
\label{sec:vertex-property-columns}

Columnar data-structures can be directly used for storing vertex properties. By  \ref{ssec:graph-schema} and \ref{ssec:structured-unstructured-properties}, we know that the structured properties of a vertex can be determined by its vertex type. Let $(t_1, t_2, ...)$ be the vertex types in the system. Then, $t_i \rightarrow (p_{i,1},  p_{i,2}, ...)$, where $p_{i, j}$ is a structured vertex property of $t_i$. Further, let $d_{i,j}$ be the data type of property $p_{i,j}$. We define a \emph{vertex property column}, $CV_{i,j}$, for each $p_{i,j}$ in the system, having a fixed data type $d_{i,j}$. Each column stores values of a single property $p_{i,j}$ of all vertices of type $t_i$ at consecutive locations. A collection of columns of a particular vertex type $t_i$ is called the column family, $FCV_i$. Structurally, a column is simply a single-dimensional array of \emph{fixed-width} elements and can be accessed by its positional offset. We store one vertex property per element. Having fixed-width elements greatly simplifies reading a values from the column in constant-time using their positional offsets.

We omit explicit ordering in columns. This is a deviation from most of the \gls{gdbms} that order their vertex properties by either \texttt{vertexID} or a property value.  We base our decision on \ref{ssec:vertices-unordered}. In the absence of an explicit order, the property values of a new vertex get \emph{appended} to the respective columns of a family, in a synchronized fashion. This induces a common natural ordering in columns of a family. Therefore, the least we guarantee is that property values of a vertex of type $t_i$ appear at the same location (having same positional offset) in respective columns of family $FCV_i$. Having no order makes insertions to the vertex property columns easy and straightforward.

A property value of vertex $v$ can be read from a columns, using 2 pieces of information, 1) property attribute $p_{i,j}$, and 3) positional offset in the column. $p_{i,j}$, stored as 4-byte values in the system, locates the column, while the positional offset locates the property in that column. Here, the property offset serves as the identifier for $v$ in the column and hence, in column's family. Alternatively, property in a column can also be accessed directly through $v$'s \texttt{vertexID}. However, using \texttt{vertexID}s for identifying locations in a column has 2 major implications:
\vspace{-6pt}
\begin{enumerate}
	\item \textbf{Storage}: \texttt{vertexID} has to be stored with each entry in the column as an identifier. We can do better by storing \texttt{vertexID} only once for a column family.
	\item \textbf{Performance}: Searching for \texttt{vertexID}s in the column to get to the property location is an $\Omega(log(n))$ operation for sorted \texttt{vertexID}s. On the other hand, positional offset accesses a property from column directly.
\end{enumerate}
\vspace{-6pt}
The \emph{common} positional offset of a vetex in a column family can be thought of as a \emph{type-level} vertex identifier. The notion can be extended to the global-level vertex indentifier - by also including vertex type information with \emph{type-level} vertex identifier. Hence, a \emph{\textbf{(vertex type, type-level positional offset)}} pair can be treated as a global identifier for a vertex in the system. It replaces 8-byte \texttt{vertexID}s. 

\vspace{-12pt}
\subparagraph{New vertex identifying scheme.}Even though the new vertex identifier takes 12-bytes (4-bytes vertex type and 8-byte positional offset), we prefer it over 8-byte \texttt{vertexID}s. The new vertex identifier makes accessing vertex properties simple and efficient. However, it takes 4-byte extra per edge to store in the adjacency lists which increases adjacency lists storage by 1.25x. We introduce storge optimizations to counter the surge in adjacency lists size in section \ref{sec:storage-optimizations}.

\section{Columns for Edge properties}
\label{sec:edge-property-columns}

Inferring from \ref{ssec:edges-ordered}, 1) the edges are ordered in the adjacency lists, 2) edges are read by the \texttt{E/I} operators in the order in which they appear in the adjacency lists, and, 3) edge properties are read in the order in which edges are read. The access pattern of edge properties calls for storing edge properties in the order of edges. However, since there are 2 ordering on the edges, we will have to store properties twice for each order. This introduces a \emph{storege-performance} tradeoff. We present two naive solutions that optimizes \emph{storage} and \texttt{performance} respectively:

\vspace{-12pt}
\subparagraph{Solution 1: (Unordered, No replication) } We use the same design as that for the vertex properties. That is, we have an unordered column for each $q_{i,j}$, where $q_{i,j}$ a structured property of edge label $l_i$. Edges in the system can be identified as \emph{(edge label, label-level positional offset)}. However, such a design would be oblivious to the fact that edges are ordered. Property readers can access the edge properties randomly. However, they cannot benefit from the cache locality which might be there lest the edge properties were ordered by the edges. As a secondary disadvantage, edges will take up 12-bytes for storing in adjacency lists, compared to 8-byte \texttt{edgeID}s.

\vspace{-12pt}
\subparagraph{Solution 2: (Ordered, Replication) } We have two columns for each $q_{i,j}$. Properties in 2 columns are ordered by the forward and backward adjacency lists respectively. This solution ensures cache locality benefits in reading properties from consecutive locations either. However, there is replication that doubles the memory footprint of edge property storage. The solution also comes with an advantageous side effect. We can completely get rid of storing 8-byte \texttt{edgeID}s in the adjacency lists. The location of an edge property in the column is simply the positional offset of the edge in the adjacency list. However, updates get tough, with each edge insertion needing 2x insertions in the edge property columns.

\subsection{List-level Edge Property Columns}

Solutions 1 and 2 present extreme cases for optimizing storage and performance. We take the middle ground. The idea is to avoid replication while not giving all of the cache locality benefits. On a course level, we associate properties of an edge with either the source or destination vertex and order the properties by edges in source vertex's forward adjacency list or destination vertex's backward adjacency list.

We mark each edge label $l$ with either \texttt{S} (for source) or \texttt{D} (for destination). Define an edge $e$ from vertex $u$ to $v$  that has label $l_i$ marked \texttt{S}. Then, each of $e$'s properties is stored in a family of columns whose structure and ordering is given by the forward adjacency list of $u$ ($e$'s source vertex). We call the forward adjacency list of $u$ as the \emph{defining adjacency list} of aforementioned columns, its family and $e$. Note that there is one column per edge property $q_{i,j}$ in the family. We call the column, \emph{list-level edge property column}. A single column, represented by $CE_{i,j,u,fwd}$, holds value for property $q_{i,j}$ and has the forward adjacency list of $u$ as it's defining adjacency list. Consequently, $CE_{i,j,u,fwd}$'s family is represented as $FCE_{i,u,fwd}$. Note that apart from $e$'s property value, property values of other edges in the defining adjacency list are also stored in $CE_{i,j,u,fwd}$. Since this column follows the ordering of edges in \emph{defining adjacency list}, the position of an edge in defining adjacency list is same as the position of that edge's properties in columns of a family. This position is called list-level property offset. By similar analogy, if $l_i$ is marked \texttt{D}, the columns will be defined as $CE_{i,j,v,bwd}$ and will be ordered by the backward adjacency list of $v$.

Storing properties in such a manner prevent replication. A property of an edge is ordered according to one of 2 adjacency lists an edge is associated with. We get sequential read of edge properties by reading edges from the \emph{defining adjacency list}. However, random access to edge properties by reading edges from the \emph{non-defining adjacency list} is still ensured.

Note that the property $q_{i,j}$ of edge $e$ having label $l_i$ can now be accessed by knowing 1) $q_{i,j}$, 2) source or destination vertex of $e$ , and 3) list-level positional offset of $e$ in its defining adjacency list. Accessing properties by \texttt{edgeID}s is not preferred owing to the proposition identical to that explained in the previous section. The list-level positional offset, used for accessing edge properties, serves as a \emph{(defining adjacency) list-level} identifier of that edge. The defining adjacency list of $e$ can itself be identified by $e$'s label and source or destination vertex. Hence, we can identify an edge globally by the \emph{\textbf{(edge label, source vertex, destination vertex, (default adjacency) list-level position offset)}} tuple. 

\vspace{-12pt}
\subparagraph{New edge identifying scheme.} The plain downside of using this new edge identification scheme is that there are a lot of components to store that costs 36 bytes $(=4+12+12+8)$ in total. However, there are trivial optimizations: 
\begin{itemize}
	\item We do not need to store the 4 components of the new edge identifier. While extending from an adjacency list, the \emph{extending} vertex identifier and the edge label is already known. They are not stored again on the list. \emph{Neighbouring} vertex identifier is stored from prior, to each edge. Therefore, the only overhead is the list-level positional offset which is equivalent to the overhead from \texttt{edgeID}s.
	\item The width of a list-level positional offset is mostly 1 or 2 bytes. This follows the fact that the number of edges in most of the adjacency list is too small (by the power-law). Hence, the size of the positional offset is variable and hence storing them as variable-width brings in a lot of memory space savings.
	\item List-level positional offsets can be omitted from storage in adjacency lists at many special cases. We look into such optimizations in \ref{sec:storage-optimizations}.
\end{itemize}

\subsection{Page-level Edge Property Columns}

List-level edge property columns is a good solution for storing edge properties but practically infeasible. This is primarily because insertion on new edges is difficult to manage and keeping a list-level edge property column family per defining adjacency list adds much pointer overhead to the system. We review the limitations of list-level edge property columns below


\vspace{-12pt}
\subparagraph{Limitations of List-level Edge Property Columns.}
Storing edge properties in list-level edge property lists are not update friendly owing to the existence of strict ordering in all the columns. For example, assume a scenario where a new edge is has to be inserted at \emph{position 5} of a defining adjacency list $A$ of size \emph{10}. Such insertions have 2 effects: 
\begin{enumerate}
	\item All the columns with $A$ as the defining adjacency list has to undergo shifting to make space for the properties of the incoming new edge at position 5.
	\item The list-level positional offsets of edges appearing at positions $\geq 5$ in $A$ gets incremented by $1$. Incrementing these edges in \emph{non-defining adjacency lists} is not a trivial task.  
\end{enumerate}

\vspace{-16pt}
\subparagraph{Page-level Edge Property Columns.}
To make insertions easy, we omit the proposition of strict ordering on columns. The high-level idea is to sacrifice sequential reads of edge properties in a column but preserve much of the cache locality.

We map $n$ list-level edge property columns $CE$ into \emph{one} \emph{page-level edge property column}, $PCE$. Hence, a single page-level edge property column holds the property of edges from $n$ defining adjacency lists. The mapping between list-level edge property column and page-level edge property column is intuitive. $CE_{i,j,v,fwd}$ maps to $PCE_{i,j,t,b,fwd}$, iff $v$ has vertex type $t$ and \emph{b = (v's type-level positional offset)/n}. $b$ stands for a \emph{bucket}ID since  one $PCE$ \emph{buckets} multiple $CE$s together. The main benefit of the page-level edge property column comes from the fact that it is unordered. Hence, new edge insertions becomes easy as they get appended into relevant page-level edge property column in a fashion similar to insertions in vertex property columns in \ref{sec:vertex-property-columns}. Instead of list-level positional offset, edge property is access by a page-level positional offset. Note that the page-level positional offset no longer refers to the position of an edge in \emph{any} of the page-level edge property column's defining adjacency lists. The page-level positional offset only refers to the position of properties of edge in page-level edge property column of a family. 

By the design of page-level edge property columns, the property values of edges in the column's defining adjacency lists is randomly distributed across the column. The randomness in the distribution of property values is a function of $n$. Larger the value of $n$, more sparsely (far-apart) the property values of edges in a defining adjacency list are distributed. For the best case of $n=1$, property values of edges in the only defining adjacency list are next to each other, unordered. Choosing an appropriate $n$ is pivotal in harnessing cache benefits. Generally, we aim to reduce the ratio of the number of L1 cache misses to the number of value access from the column. This is desirable when the number of value accesses is less. 

The driving goal is to choose $n$ such that property values of edges in a defining adjacency list are not too sparse for a cache miss to occur with \emph{each} access of a value. The value of $n$ depends on a number of factors: 
\begin{enumerate}
	\item The average number of edges in the adjacency lists.
	\item  
\end{enumerate}

Hence, we fix a value of $n$ such that 1) the number of L1 cache misses is less than the number of accesses to the column, and 2) the size of the page-level edge property column can fit comfortably in low levels of cache.

The value of $n$ is chosen heuristically for the system such that, a page-level edge property column can comfortably fit in the L1 cache. The typical value of n lies in the range between $[32, 512]$. The value of $n$ depends on a number of factors: 
\begin{enumerate}
	\item The average number of edges in the adjacency lists.
	\item  
\end{enumerate} 


For example, let $n$ be 6 and the average number of edges in an adjacency list be 8, then a page-level edge property column with 4-byte values takes 192-bytes. Assuming the L1 cache line size to be 64-bytes, the column will fit in 3 cache lines. This means, at the worst case, there will be 3 cache mises in reading a property of edges from a particular adjacency list.




\section{Storing edges in adjacency lists}
\label{sec:adjacency-lists}

\section{Storage optimizations}
\label{sec:storage-optimizations}

In this section, we present storage optimizations that can be applied on our data-structures and new identification schemes. These optimizations are critical in reducing the overall memory footprint as well as in query performance. 

\subsection{Vertex property columns for single multiplicity edge labels.}

Single multiplicity edge labels are edge labels that have one of the following cardinalities, 1) 1..1 (one-to-one), 2) 1..n (one-to-many), and 3) n..1 (many-to-one). Such labels guarantee that there will be at most a single edge from the source vertex or to the destination vertex. For instance, let us consider an edge label BIRTHDAY that connects a vertex of type PERSON to the vertex of type DATE. The cardinality of label BIRTHDAY is \emph{many-to-one}. That is, a vertex PERSON is connected to only one vertex of type DATE, while a vertex DATE can have connections to multiple PERSONs through the edge of label BIRTHDAY.



\subsection{Removing unnecessary edge information.}

\subsection{Fixed-length null suppression.}


\chapter{Columnar Compression}
\label{columnar-compression}

Compression in column-oriented storage has been extensively researched \cite{abadi-col-comp, abadi-sparse-col} in the \gls{rdbms} setting. 

\include{list-based processing}

\include{zone maps}

\include{Evaluation}

\include{Related Work}

\include{Conclusion and Future Work}

% B I B L I O G R A P H Y
% -----------------------

% The following statement selects the style to use for references.  It controls the sort order of the entries in the bibliography and also the formatting for the in-text labels.
\bibliographystyle{plain}
% This specifies the location of the file containing the bibliographic information.  
% It assumes you're using BibTeX (if not, why not?).
\cleardoublepage % This is needed if the book class is used, to place the anchor in the correct page,
                 % because the bibliography will start on its own page.
                 % Use \clearpage instead if the document class uses the "oneside" argument
\phantomsection  % With hyperref package, enables hyperlinking from the table of contents to bibliography             
% The following statement causes the title "References" to be used for the bibliography section:
\renewcommand*{\bibname}{References}

% Add the References to the Table of Contents
\addcontentsline{toc}{chapter}{\textbf{References}}

\bibliography{ref}
% Tip 5: You can create multiple .bib files to organize your references. 
% Just list them all in the \bibliogaphy command, separated by commas (no spaces).

% The following statement causes the specified references to be added to the bibliography% even if they were not 
% cited in the text. The asterisk is a wildcard that causes all entries in the bibliographic database to be included (optional).
\nocite{*}

% The \appendix statement indicates the beginning of the appendices.
\appendix
% Add a title page before the appendices and a line in the Table of Contents
\chapter*{APPENDICES}
\addcontentsline{toc}{chapter}{APPENDICES}
%======================================================================
\chapter[PDF Plots From Matlab]{Matlab Code for Making a PDF Plot}
\label{AppendixA}
% Tip 4: Example of how to get a shorter chapter title for the Table of Contents 
%======================================================================
\section{Using the GUI}
Properties of Matab plots can be adjusted from the plot window via a graphical interface. Under the Desktop menu in the Figure window, select the Property Editor. You may also want to check the Plot Browser and Figure Palette for more tools. To adjust properties of the axes, look under the Edit menu and select Axes Properties.

To set the figure size and to save as PDF or other file formats, click the Export Setup button in the figure Property Editor.

\section{From the Command Line} 
All figure properties can also be manipulated from the command line. Here's an example: 
\begin{verbatim}
x=[0:0.1:pi];
hold on % Plot multiple traces on one figure
plot(x,sin(x))
plot(x,cos(x),'--r')
plot(x,tan(x),'.-g')
title('Some Trig Functions Over 0 to \pi') % Note LaTeX markup!
legend('{\it sin}(x)','{\it cos}(x)','{\it tan}(x)')
hold off
set(gca,'Ylim',[-3 3]) % Adjust Y limits of "current axes"
set(gcf,'Units','inches') % Set figure size units of "current figure"
set(gcf,'Position',[0,0,6,4]) % Set figure width (6 in.) and height (4 in.)
cd n:\thesis\plots % Select where to save
print -dpdf plot.pdf % Save as PDF
\end{verbatim}

\end{document}
